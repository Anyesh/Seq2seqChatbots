\documentclass[11pt]{article}
\usepackage{times}
\usepackage{url}
\usepackage{latexsym}
\usepackage[utf8]{inputenc}
\usepackage{color}
\usepackage{multirow}
%\usepackage{latexsym}
\usepackage{booktabs,amsmath,multicol}
%\usepackage{hyperref}
% really have to

\title{Title}
\author{ % G\'abor Recski and Eszter Ikl\'odi and Katalin Pajkossy
    Rich\'ard Cs\'aky \\
    Department of Automation and Applied Informatics \\
    Budapest University of Technology and Economics \\
    {\tt ricsinaruto@hotmail.com} \\
    }
\date{}

\begin{document}

\maketitle

\begin{abstract}
Absztrakt
\end{abstract}

\begin{abstract}
Abstract
\end{abstract}

\section{Introduction}
\label{sec:intro}

A chatbotokat már a régi görögök is...

Ez a cikk...
    background
        - the task
        - seq2seq
        - ...
    critical survey of literature
    experiments with transformer model on various datasets
    directions for future work


\section{History of chatbots}
\label{sec:background}

    \subsection{Task}
    Mik azok a chatbotok
    
    Miben mas mint a task-specific dolgok
    
    ...


    \subsection{Early approaches}

    \subsection{The seq2seq model}


\section{Background}

    \subsection{Recent chatbot architectures}
75+ cikk...
    \cite{Cho:2014}
    itt is kellenek section-ok (75 cikk csoportositva)
    pl. \subsubsection{Persona}
    - csoportonkent is kell kis osszefoglalo a problemakrol (1-1)

    \subsection{Criticism}
        \subsubsection{Evaluation}
        ...

    \subsection{Summary}
    

\section{Experiments}
\label{sec:experiments}

\subsection{The Tranformer model}
\label{sec:transformer}

\subsection{Datasets}



\bibliographystyle{plain}
\bibliography{chatbots}
\end{document}
